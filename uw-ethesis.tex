%======================================================================
% University of Waterloo Thesis Template for LaTeX 
% Last Updated August 21, 2018 
% by Stephen Carr, IST Client Services, 
% University of Waterloo, 200 University Ave. W., Waterloo, Ontario, Canada
% FOR ASSISTANCE, please send mail to rt-IST-CSmathsci@rt.uwaterloo.ca

% DISCLAIMER
% To the best of our knowledge, this template satisfies the current uWaterloo thesis requirements.
% However, it is your responsibility to assure that you have met all 
% requirements of the University and your particular department.

% Many thanks for the feedback from many graduates who assisted the development of this template.
% Also note that there are explanatory comments and tips throughout this template.
%======================================================================
% Some important notes on using this template and making it your own...

% The University of Waterloo has required electronic thesis submission since October 2006. 
% See the uWaterloo thesis regulations at
% https://uwaterloo.ca/graduate-studies/thesis.
% This thesis template is geared towards generating a PDF 
% version optimized for viewing on an electronic display, including 
% hyperlinks within the PDF.

% DON'T FORGET TO ADD YOUR OWN NAME AND TITLE in the "hyperref" package
% configuration below. THIS INFORMATION GETS EMBEDDED IN THE PDF FINAL PDF DOCUMENT.
% You can view the information if you view properties of the PDF document.

% Many faculties/departments also require one or more printed
% copies. This template attempts to satisfy both types of output. See additional notes below.
% It is based on the standard "book" document class which provides all necessary 
% sectioning structures and allows multi-part theses.

% If you are using this template in Overleaf (cloud-based collaboration service), then it is 
% automatically processed and previewed for you as you edit.

% For people who prefer to install their own LaTeX distributions on their own computers, and process 
% the source files manually, the following notes provide the sequence of tasks:
 
% E.g. to process a thesis called "mythesis.tex" based on this template, run:

% pdflatex mythesis	-- first pass of the pdflatex processor
% bibtex mythesis	-- generates bibliography from .bib data file(s)
% makeindex         -- should be run only if an index is used 
% pdflatex mythesis	-- fixes numbering in cross-references, bibliographic references, glossaries, index, etc.
% pdflatex mythesis	-- it takes a couple of passes to completely process all cross-references

% If you use the recommended LaTeX editor, Texmaker, you would open the mythesis.tex
% file, then click the PDFLaTeX button. Then run BibTeX (under the Tools menu).
% Then click the PDFLaTeX button two more times. If you have an index as well,
% you'll need to run MakeIndex from the Tools menu as well, before running pdflatex
% the last two times.

% N.B. The "pdftex" program allows graphics in the following formats to be
% included with the "\includegraphics" command: PNG, PDF, JPEG, TIFF
% Tip 1: Generate your figures and photos in the size you want them to appear
% in your thesis, rather than scaling them with \includegraphics options.
% Tip 2: Any drawings you do should be in scalable vector graphic formats:
% SVG, PNG, WMF, EPS and then converted to PNG or PDF, so they are scalable in
% the final PDF as well.
% Tip 3: Photographs should be cropped and compressed so as not to be too large.

% To create a PDF output that is optimized for double-sided printing: 
%
% 1) comment-out the \documentclass statement in the preamble below, and
% un-comment the second \documentclass line.
%
% 2) change the value assigned below to the boolean variable
% "PrintVersion" from "false" to "true".

%======================================================================
%   D O C U M E N T   P R E A M B L E
% Specify the document class, default style attributes, and page dimensions, etc.
% For hyperlinked PDF, suitable for viewing on a computer, use this:
\documentclass[letterpaper,12pt,titlepage,oneside,final]{book}
 
% For PDF, suitable for double-sided printing, change the PrintVersion variable below
% to "true" and use this \documentclass line instead of the one above:
%\documentclass[letterpaper,12pt,titlepage,openright,twoside,final]{book}

% Some LaTeX commands I define for my own nomenclature.
% If you have to, it's easier to make changes to nomenclature once here than in a 
% million places throughout your thesis!
\newcommand{\package}[1]{\textbf{#1}} % package names in bold text
\newcommand{\cmmd}[1]{\textbackslash\texttt{#1}} % command name in tt font 
\newcommand{\href}[1]{#1} % does nothing, but defines the command so the
    % print-optimized version will ignore \href tags (redefined by hyperref pkg).
%\newcommand{\texorpdfstring}[2]{#1} % does nothing, but defines the command
% Anything defined here may be redefined by packages added below...

% This package allows if-then-else control structures.
\usepackage{ifthen}
\newboolean{PrintVersion}
\setboolean{PrintVersion}{false}
% CHANGE THIS VALUE TO "true" as necessary, to improve printed results for hard copies
% by overriding some options of the hyperref package, called below.

%\usepackage{nomencl} % For a nomenclature (optional; available from ctan.org)
\usepackage{amsmath,amssymb,amstext} % Lots of math symbols and environments
\usepackage[pdftex]{graphicx} % For including graphics N.B. pdftex graphics driver 

% Hyperlinks make it very easy to navigate an electronic document.
% In addition, this is where you should specify the thesis title
% and author as they appear in the properties of the PDF document.
% Use the "hyperref" package 
% N.B. HYPERREF MUST BE THE LAST PACKAGE LOADED; ADD ADDITIONAL PKGS ABOVE
\usepackage[pdftex,pagebackref=false]{hyperref} % with basic options
%\usepackage[pdftex,pagebackref=true]{hyperref}
		% N.B. pagebackref=true provides links back from the References to the body text. This can cause trouble for printing.
\hypersetup{
    plainpages=false,       % needed if Roman numbers in frontpages
    unicode=false,          % non-Latin characters in Acrobat’s bookmarks
    pdftoolbar=true,        % show Acrobat’s toolbar?
    pdfmenubar=true,        % show Acrobat’s menu?
    pdffitwindow=false,     % window fit to page when opened
    pdfstartview={FitH},    % fits the width of the page to the window
%    pdftitle={uWaterloo\ LaTeX\ Thesis\ Template},    % title: CHANGE THIS TEXT!
%    pdfauthor={Author},    % author: CHANGE THIS TEXT! and uncomment this line
%    pdfsubject={Subject},  % subject: CHANGE THIS TEXT! and uncomment this line
%    pdfkeywords={keyword1} {key2} {key3}, % list of keywords, and uncomment this line if desired
    pdfnewwindow=true,      % links in new window
    colorlinks=true,        % false: boxed links; true: colored links
    linkcolor=blue,         % color of internal links
    citecolor=green,        % color of links to bibliography
    filecolor=magenta,      % color of file links
    urlcolor=cyan           % color of external links
}
\ifthenelse{\boolean{PrintVersion}}{   % for improved print quality, change some hyperref options
\hypersetup{	% override some previously defined hyperref options
%    colorlinks,%
    citecolor=black,%
    filecolor=black,%
    linkcolor=black,%
    urlcolor=black}
}{} % end of ifthenelse (no else)

\usepackage[automake,toc,abbreviations]{glossaries-extra} % Exception to the rule of hyperref being the last add-on package
% If glossaries-extra is not in your LaTeX distribution, get it from CTAN (http://ctan.org/pkg/glossaries-extra), 
% although it's supposed to be in both the TeX Live and MikTeX distributions. There are also documentation and 
% installation instructions there.

% Setting up the page margins...
% uWaterloo thesis requirements specify a minimum of 1 inch (72pt) margin at the
% top, bottom, and outside page edges and a 1.125 in. (81pt) gutter
% margin (on binding side). While this is not an issue for electronic
% viewing, a PDF may be printed, and so we have the same page layout for
% both printed and electronic versions, we leave the gutter margin in.
% Set margins to minimum permitted by uWaterloo thesis regulations:
\setlength{\marginparwidth}{0pt} % width of margin notes
% N.B. If margin notes are used, you must adjust \textwidth, \marginparwidth
% and \marginparsep so that the space left between the margin notes and page
% edge is less than 15 mm (0.6 in.)
\setlength{\marginparsep}{0pt} % width of space between body text and margin notes
\setlength{\evensidemargin}{0.125in} % Adds 1/8 in. to binding side of all 
% even-numbered pages when the "twoside" printing option is selected
\setlength{\oddsidemargin}{0.125in} % Adds 1/8 in. to the left of all pages
% when "oneside" printing is selected, and to the left of all odd-numbered
% pages when "twoside" printing is selected
\setlength{\textwidth}{6.375in} % assuming US letter paper (8.5 in. x 11 in.) and 
% side margins as above
\raggedbottom

% The following statement specifies the amount of space between
% paragraphs. Other reasonable specifications are \bigskipamount and \smallskipamount.
\setlength{\parskip}{\medskipamount}

% The following statement controls the line spacing.  The default
% spacing corresponds to good typographic conventions and only slight
% changes (e.g., perhaps "1.2"), if any, should be made.
\renewcommand{\baselinestretch}{1} % this is the default line space setting

% By default, each chapter will start on a recto (right-hand side)
% page.  We also force each section of the front pages to start on 
% a recto page by inserting \cleardoublepage commands.
% In many cases, this will require that the verso (left-hand) page be
% blank, and while it should be counted, a page number should not be
% printed.  The following statements ensure a page number is not
% printed on an otherwise blank verso page.
\let\origdoublepage\cleardoublepage
\newcommand{\clearemptydoublepage}{%
  \clearpage{\pagestyle{empty}\origdoublepage}}
\let\cleardoublepage\clearemptydoublepage

% Define Glossary terms (This is properly done here, in the preamble and could also be \input{} from a separate file...)
% Main glossary entries -- definitions of relevant terminology
\newglossaryentry{computer}
{
name=computer,
description={A programmable machine that receives input data,
               stores and manipulates the data, and provides
               formatted output}
}

% Nomenclature glossary entries -- New definitions, or unusual terminology
\newglossary*{nomenclature}{Nomenclature}
\newglossaryentry{dingledorf}
{
type=nomenclature,
name=dingledorf,
description={A person of supposed average intelligence who makes incredibly brainless misjudgments}
}

% List of Abbreviations (abbreviations type is built in to the glossaries-extra package)
\newabbreviation{aaaaz}{AAAAZ}{American Association of Amateur Astronomers and Zoologists}

% List of Symbols
\newglossary*{symbols}{List of Symbols}
\newglossaryentry{rvec}
{
name={$\mathbf{v}$},
sort={label},
type=symbols,
description={Random vector: a location in n-dimensional Cartesian space, where each dimensional component is determined by a random process}
}
 
\makeglossaries

%======================================================================
%   L O G I C A L    D O C U M E N T
% The logical document contains the main content of your thesis.
% Being a large document, it is a good idea to divide your thesis
% into several files, each one containing one chapter or other significant 
% chunk of content, so you can easily shuffle things around later if desired.
%======================================================================
\begin{document}

%----------------------------------------------------------------------
% FRONT MATERIAL
% title page,declaration, borrowers' page, abstract, acknowledgements,
% dedication, table of contents, list of tables, list of figures, nomenclature, etc.
%----------------------------------------------------------------------
% T I T L E   P A G E
% -------------------
% Last updated June 14, 2017, by Stephen Carr, IST-Client Services
% The title page is counted as page `i' but we need to suppress the
% page number. Also, we don't want any headers or footers.
\pagestyle{empty}
\pagenumbering{roman}

% The contents of the title page are specified in the "titlepage"
% environment.
\begin{titlepage}
        \begin{center}
        \vspace*{1.0cm}

        \Huge
        {\bf University of Waterloo E-Thesis Template for \LaTeX }

        \vspace*{1.0cm}

        \normalsize
        by \\

        \vspace*{1.0cm}

        \Large
        Zeynep Akkalyoncu Yilmaz \\

        \vspace*{3.0cm}

        \normalsize
        A thesis \\
        presented to the University of Waterloo \\ 
        in fulfillment of the \\
        thesis requirement for the degree of \\
        Master of Mathematics \\
        in \\
        Computer Science \\

        \vspace*{2.0cm}

        Waterloo, Ontario, Canada, 2019 \\

        \vspace*{1.0cm}

        \copyright\ Zeynep Akkalyoncu Yilmaz 2019 \\
        \end{center}
\end{titlepage}

% The rest of the front pages should contain no headers and be numbered using Roman numerals starting with `ii'
\pagestyle{plain}
\setcounter{page}{2}

\cleardoublepage % Ends the current page and causes all figures and tables that have so far appeared in the input to be printed.
% In a two-sided printing style, it also makes the next page a right-hand (odd-numbered) page, producing a blank page if necessary.

 
% E X A M I N I N G   C O M M I T T E E (Required for Ph.D. theses only)
% Remove or comment out the lines below to remove this page
%\begin{center}\textbf{Examining Committee Membership}\end{center}
%  \noindent
%The following served on the Examining Committee for this thesis. The decision of the Examining Committee is by majority vote.
%  \bigskip
%  
%  \noindent
%\begin{tabbing}
%Internal-External Member: \=  \kill % using longest text to define tab length
%External Examiner: \>  Bruce Bruce \\ 
%\> Professor, Dept. of Philosophy of Zoology, University of Wallamaloo \\
%\end{tabbing} 
%  \bigskip
%  
%  \noindent
%\begin{tabbing}
%Internal-External Member: \=  \kill % using longest text to define tab length
%Supervisor(s): \> Doris Johnson \\
%\> Professor, Dept. of Zoology, University of Waterloo \\
%\> Andrea Anaconda \\
%\> Professor Emeritus, Dept. of Zoology, University of Waterloo \\
%\end{tabbing}
%  \bigskip
%  
%  \noindent
%  \begin{tabbing}
%Internal-External Member: \=  \kill % using longest text to define tab length
%Internal Member: \> Pamela Python \\
%\> Professor, Dept. of Zoology, University of Waterloo \\
%\end{tabbing}
%  \bigskip
%  
%  \noindent
%\begin{tabbing}
%Internal-External Member: \=  \kill % using longest text to define tab length
%Internal-External Member: \> Deepa Thotta \\
%\> Professor, Dept. of Philosophy, University of Waterloo \\
%\end{tabbing}
%  \bigskip
%  
%  \noindent
%\begin{tabbing}
%Internal-External Member: \=  \kill % using longest text to define tab length
%Other Member(s): \> Leeping Fang \\
%\> Professor, Dept. of Fine Art, University of Waterloo \\
%\end{tabbing}

\cleardoublepage

% D E C L A R A T I O N   P A G E
% -------------------------------
  % The following is a sample Delaration Page as provided by the GSO
  % December 13th, 2006.  It is designed for an electronic thesis.
%  \noindent
%I hereby declare that I am the sole author of this thesis. This is a true copy of the thesis, including any required final revisions, as accepted by my examiners.
%
%  \bigskip
%  
%  \noindent
%I understand that my thesis may be made electronically available to the public.

\cleardoublepage

% A B S T R A C T
% ---------------

\begin{center}\textbf{Abstract}\end{center}
Standard bag-of-words term-matching techniques in document retrieval fail to exploit rich semantic information embedded in the document texts.
One promising recent trend in facilitating context-aware semantic matching has been the development of massively pretrained language models, culminating in BERT as its most popular example today.
In this work, we propose adapting BERT as a neural reranker for document retrieval with large improvements on news articles.
Two fundamental issues arise in applying BERT to ``ad hoc'' document retrieval on newswire collections:\
relevance judgements in existing test collections are provided only at the document level, and documents often exceed the length that BERT was designed to handle.
To overcome these challenges, we compute and aggregate sentence-level relevance scores to rank documents.
We solve the problem of lack of appropriate relevance judgements by leveraging sentence-level and passage-level relevance judgements available in collections from other domains to capture cross-domain notions of relevance.
We demonstrate that models of relevance can be transferred across domains.
By leveraging semantic cues learned across various domains, we propose a model that achieves state-of-the-art results across three standard TREC newswire collections.
We explore the effects of cross-domain relevance transfer, and trade-offs between using document and sentence scores for document ranking.
We also present an end-to-end document retrieval system that incorporates the open-source Anserini information retrieval toolkit, discussing the related technical challenges and design decisions.

\cleardoublepage

% A C K N O W L E D G E M E N T S
% -------------------------------

\begin{center}\textbf{Acknowledgements}\end{center}

I would like to thank all the little people who made this thesis possible.
\cleardoublepage

% D E D I C A T I O N
% -------------------

\begin{center}\textbf{Dedication}\end{center}

This is dedicated to the one I love.
\cleardoublepage

% T A B L E   O F   C O N T E N T S
% ---------------------------------
\renewcommand\contentsname{Table of Contents}
\tableofcontents
\cleardoublepage
\phantomsection    % allows hyperref to link to the correct page

% L I S T   O F   T A B L E S
% ---------------------------
\addcontentsline{toc}{chapter}{List of Tables}
\listoftables
\cleardoublepage
\phantomsection		% allows hyperref to link to the correct page

% L I S T   O F   F I G U R E S
% -----------------------------
\addcontentsline{toc}{chapter}{List of Figures}
\listoffigures
\cleardoublepage
\phantomsection		% allows hyperref to link to the correct page

% GLOSSARIES (Lists of definitions, abbreviations, symbols, etc. provided by the glossaries-extra package)
% -----------------------------
\printglossaries
\cleardoublepage
\phantomsection		% allows hyperref to link to the correct page

% Change page numbering back to Arabic numerals
\pagenumbering{arabic}

 

%----------------------------------------------------------------------
% MAIN BODY
% We suggest using a separate file for each chapter of your thesis.
% Start each chapter file with the \chapter command.
% Only use \documentclass or \begin{document} and \end{document} commands 
% in this master document.
% Tip 4: Putting each sentence on a new line is a way to simplify later editing.
%----------------------------------------------------------------------
%======================================================================
\chapter{Introduction}
%======================================================================

Document retrieval refers to the task of generating a ranking of documents from a large corpus $ D $ in response to a query $ Q $.
In a typical document retrieval pipeline, an inverted index is constructed in advance from the collection, which often comprises unstructured text documents, for fast access during retrieval.
When the user issues a query, the query representation is matched against the index, computing a similarity score for each document.
The top most relevant documents based on their closeness to the query are returned to the user in order of relevance.
This procedure may be followed by a subsequent re-ranking stage where the candidate documents outputted by the previous step are further re-ranked in a way that maximizes some retrieval metric such as average precision (AP).

\begin{figure}[b!]
	\begin{framed}
		\centering
    		\textbf{Query:} international art crime \\
    		\textbf{Text:} The thieves demand a ransom of \$2.2 million for the works and return one of them.
	\end{framed}
\label{query-sent-example}
 \caption{An example of a query-text pair from the TREC Robust04 collection where a relevant piece of text does not contain direct query matches.}
\end{figure}

Document retrieval systems traditionally rely on term-matching techniques, such as BM25, to judge the relevance of documents in a corpus.
More specifically, the more common terms a document shares with the query, the more relevant it is considered.
As a result, these systems may fail to detect documents that do not contain the exact query terms, but are nonetheless relevant.
For example, consider a document that expresses relevant information in a way that cannot be resolved without the use of external semantic tools.
Figure \ref{query-sent-example} displays one such query-text pair where words semantically close to the query need to be identified to establish relevance.
This ``vocabulary mismatch'' problem represents a long-standing challenge in information retrieval.
To put its significance into context, Zhao et al. \cite{zhao2010term} show that the average query terms may not appear in as many as 40\% of relevant documents in TREC ``ad hoc'' retrieval datasets.

Clearly, the classic exact matching approach to document retrieval neglects to exploit rich semantic information embedded in the documents.
To overcome this shortcoming, a number of models such as Latent Semantic Analysis \cite{deerwester1990indexing} that maps both queries and documents into distributed representations has been proposed.
This innovation has enabled semantic matching to aid in document retrieval by extracting useful semantic matching signals.
With the advent of neural networks, neural language models have been quickly adopted to learn better distributed representations of text.
\myworries{How much better? Why?}
Moreover, deep neural models have eliminated the need to manually engineer natural language features.
Therefore, deep neural networks have since largely replaced the earlier models based on manual decomposition of document matrices.
\myworries{Examples...}

%Despite growing interest in neural networks, some researchers have recently voiced concern as to whether their use has truly contributed to progress in the field of information retrieval \myworries{citation}.
%\myworries{Take stuff from their neural hype paper}

One recent innovation that has changed the tide in NLP research has been massively pre-trained language models with its most popular example today being Bidirectional Encoder Representation Transformers (BERT) \cite{devlin2018bert}.
BERT has achieved state-of-the-art results across a wide range of NLP tasks from question answering to machine translation.
While BERT has enjoyed widespread adoption across the NLP community, its application in information retrieval research has been limited in comparison.
Guo et al. \cite{guo2016deep} suggest that the lackluster success of deep neural networks in information retrieval may be owing to the fact that crucial characteristics of the ``ad hoc'' document retrieval task are not properly addressed.
Specifically, they emphasize that the relevance matching problem in information retrieval and semantic matching problem in natural language processing are fundamentally different in that the former depends heavily on exact matching signals, query term importance and diverse matching requirements.
In other words, it is crucial to strike a good balance between exact and semantic matching in document retrieval.
For this reason, neural models are usually involved in multi-stage architectures where a list of candidate documents are retrieved with a standard bag-of-words term-matching technique as described above.
The documents in this list are then rescored and reranked by the neural model.
\myworries{Some notable examples include...}

%%%

In this thesis, we present a novel way to apply BERT to ``ad hoc'' document retrieval on long documents -- particularly, newswire articles.
A BERT reranker is deployed as part of an end-to-end document retrieval pipeline with significant improvements on standard TREC newswire collections.
Specifically, we adapt BERT for binary relevance classification over text to capture notions of relevance.
\myworries{One more sentence to describe?}
We point out that applying BERT to document retrieval on newswire documents is not trivial due to two principal challenges.
Firist of all, BERT has a maximum input length of 512 tokens, which is insufficient to accommodate the entirety of most news articles.
To put this into perspective, a typical TREC Robust04 document has a median length of 679 tokens, and in fact, 66\% of all documents are longer than 512 tokens.
Secondly, most collections provide relevance judgements only at the document level.
Therefore, we only know what documents are relevant for a given query, but not the specific spans within the document.
To further aggravate this issue, a document is considered relevant as long as some part of it is relevant, and most of the document often has nothing to do with the query.

We address the abovementioned challenges by proposing two effective innovations:\
First, instead of relying solely on document-level relevance judgements, we aggregate sentence-level evidence to rank documents.
As mentioned before, since standard newswire collections lack sentence level judgements to facilitate this approach, we instead explore leveraging sentence-level or passage-level judgements already available in collections in other domains, such as tweets and reading comprehension.
To this end, we fine-tune BERT models on these collections to learn models of relevance.
Surprisingly, we demonstrate that models of relevance can indeed be successfully transferred across domains.
%We are able to use BERT models trained on out-of-domain collections on newswire documents to compute a relevance score for each constituent sentence.
It is important to note that the representational power of neural networks come at the cost of challenges in interpretability.
For this reason, we dedicate a portion of this thesis to error analysis experiments in an attempt to qualify and better understand the cross-domain transfer effects.
We also elaborate on the challenges encountered in the implementation of such an end-to-end retrieval pipeline in an attempt to bridge the worlds of natural language processing and information retrieval from a software engineering perspective.

\section{Contributions}

The main contributions of this thesis can be summarized as follows:\\

\begin{itemize}
\item
We present two innovations to successfully apply BERT to \textit{ad hoc} document retrieval with large improvements:\
integrating sentence-level evidence to address the fact that BERT cannot process long spans posed by newswire documents, and exploiting cross-domain models of relevance for collections without sentence- or passage-level annotations.
\item
We explore through various error analysis experiments on the effects of cross-domain relevance transfer with BERT as well as the contributions of BM25 and sentence scores to the final document ranking.
\item 
With the proposed model, we establish state-of-the-art effectiveness on three standard TREC newswire collections at the time of writing.\
\myworries{neural or otherwise}
\item
We release an end-to-end pipeline that applies BERT to document retrieval over large document collections via integration with the open-source Anserini information retrieval toolkit.
We elaborate on the technical challenges in the integration of NLP and IR capabilities, along with the design rationale behind our approach to tightly-coupled integration between Python to support neural networks and the Java Virtual Machine to support document retrieval using the open-source Lucene search library.
\myworries{something about demo, TREC DL...}

\end{itemize}

\section{Thesis Organization}

The remainder of this thesis is organized in the following order:\
\myworries{add link to actual chapters}
Chapter 2 reviews related work in neural document retrieval, particularly applications of BERT to document retrieval.
Chapter 3 motivates the approach with some background information on the task, and introduces the datasets used for both training and evaluation as well as metrics.
Chapter 4 proposes an end-to-end pipeline for document retrieval with BERT by elaborating on the design decisions and challenges.
\myworries{What about TREC DL? MS MARCO?}
Chapter 5 describes the experimental setup, and presents the results on three newswire collections -- Robust04, Core17 and Core18.
Chapter 6 concludes the thesis by summarizing the contributions and discussing future work.

%In the beginning, there was $\pi$:
%
%\begin{equation}
%   e^{\pi i} + 1 = 0  \label{eqn_pi}
%\end{equation}
%A \gls{computer} could compute $\pi$ all day long. In fact, subsets of digits of $\pi$'s decimal approximation would make a good source for psuedo-random vectors, \gls{rvec} . 
%
%%----------------------------------------------------------------------
%\section{State of the Art}
%%----------------------------------------------------------------------
%
%See equation \ref{eqn_pi} on page \pageref{eqn_pi}.\footnote{A famous equation.}
%
%\section{Some Meaningless Stuff}
%
%The credo of the \gls{aaaaz} was, for several years, several paragraphs of gibberish, until the \gls{dingledorf} responsible for the \gls{aaaaz} Web site realized his mistake:
%
%"Velit dolor illum facilisis zzril ipsum, augue odio, accumsan ea augue molestie lobortis zzril laoreet ex ad, adipiscing nulla. Veniam dolore, vel te in dolor te, feugait dolore ex vel erat duis nostrud diam commodo ad eu in consequat esse in ut wisi. Consectetuer dolore feugiat wisi eum dignissim tincidunt vel, nostrud, at vulputate eum euismod, diam minim eros consequat lorem aliquam et ad. Feugait illum sit suscipit ut, tation in dolore euismod et iusto nulla amet wisi odio quis nisl feugiat adipiscing luptatum minim nisl, quis, erat, dolore. Elit quis sit dolor veniam blandit ullamcorper ex, vero nonummy, duis exerci delenit ullamcorper at feugiat ullamcorper, ullamcorper elit vulputate iusto esse luptatum duis autem. Nulla nulla qui, te praesent et at nisl ut in consequat blandit vel augue ut.
%
%Illum suscipit delenit commodo augue exerci magna veniam hendrerit dignissim duis ut feugait amet dolor dolor suscipit iriure veniam. Vel quis enim vulputate nulla facilisis volutpat vel in, suscipit facilisis dolore ut veniam, duis facilisi wisi nulla aliquip vero praesent nibh molestie consectetuer nulla. Wisi nibh exerci hendrerit consequat, nostrud lobortis ut praesent dignissim tincidunt enim eum accumsan. Lorem, nonummy duis iriure autem feugait praesent, duis, accumsan tation enim facilisi qui te dolore magna velit, iusto esse eu, zzril. Feugiat enim zzril, te vel illum, lobortis ut tation, elit luptatum ipsum, aliquam dolor sed. Ex consectetuer aliquip in, tation delenit dignissim accumsan consequat, vero, et ad eu velit ut duis ea ea odio.
%
%Vero qui, te praesent et at nisl ut in consequat blandit vel augue ut dolor illum facilisis zzril ipsum. Exerci odio, accumsan ea augue molestie lobortis zzril laoreet ex ad, adipiscing nulla, et dolore, vel te in dolor te, feugait dolore ex vel erat duis. Ut diam commodo ad eu in consequat esse in ut wisi aliquip dolore feugiat wisi eum dignissim tincidunt vel, nostrud. Ut vulputate eum euismod, diam minim eros consequat lorem aliquam et ad luptatum illum sit suscipit ut, tation in dolore euismod et iusto nulla. Iusto wisi odio quis nisl feugiat adipiscing luptatum minim. Illum, quis, erat, dolore qui quis sit dolor veniam blandit ullamcorper ex, vero nonummy, duis exerci delenit ullamcorper at feugiat. Et, ullamcorper elit vulputate iusto esse luptatum duis autem esse nulla qui.
%
%Praesent dolore et, delenit, laoreet dolore sed eros hendrerit consequat lobortis. Dolor nulla suscipit delenit commodo augue exerci magna veniam hendrerit dignissim duis ut feugait amet. Ad dolor suscipit iriure veniam blandit quis enim vulputate nulla facilisis volutpat vel in. Erat facilisis dolore ut veniam, duis facilisi wisi nulla aliquip vero praesent nibh molestie consectetuer nulla, iriure nibh exerci hendrerit. Vel, nostrud lobortis ut praesent dignissim tincidunt enim eum accumsan ea, nonummy duis. Ad autem feugait praesent, duis, accumsan tation enim facilisi qui te dolore magna velit, iusto esse eu, zzril vel enim zzril, te. Nisl illum, lobortis ut tation, elit luptatum ipsum, aliquam dolor sed minim consectetuer aliquip.
%
%Tation exerci delenit ullamcorper at feugiat ullamcorper, ullamcorper elit vulputate iusto esse luptatum duis autem esse nulla qui. Volutpat praesent et at nisl ut in consequat blandit vel augue ut dolor illum facilisis zzril ipsum, augue odio, accumsan ea augue molestie lobortis zzril laoreet. Ex duis, te velit illum odio, nisl qui consequat aliquip qui blandit hendrerit. Ea dolor nonummy ullamcorper nulla lorem tation laoreet in ea, ullamcorper vel consequat zzril delenit quis dignissim, vulputate tincidunt ut."
%======================================================================
\chapter{Observations}
%======================================================================

This would be a good place for some figures and tables.

Some notes on figures and photographs\ldots

\begin{itemize}
\item A well-prepared PDF should be 
  \begin{enumerate}
    \item Of reasonable size, {\it i.e.} photos cropped and compressed.
    \item Scalable, to allow enlargment of text and drawings. 
  \end{enumerate} 
\item Photos must be bit maps, and so are not scaleable by definition. TIFF and
BMP are uncompressed formats, while JPEG is compressed. Most photos can be
compressed without losing their illustrative value.
\item Drawings that you make should be scalable vector graphics, \emph{not} 
bit maps. Some scalable vector file formats are: EPS, SVG, PNG, WMF. These can
all be converted into PNG or PDF, that pdflatex recognizes. Your drawing 
package probably can export to one of these formats directly. Otherwise, a 
common procedure is to print-to-file through a Postscript printer driver to 
create a PS file, then convert that to EPS (encapsulated PS, which has a 
bounding box to describe its exact size rather than a whole page). 
Programs such as GSView (a Ghostscript GUI) can create both EPS and PDF from PS files.
Appendix~\ref{AppendixA} shows how to generate properly sized Matlab plots and save them as PDF.
\item It's important to crop your photos and draw your figures to the size that
you want to appear in your thesis. Scaling photos with the 
includegraphics command will cause loss of resolution. And scaling down 
drawings may cause any text annotations to become too small.
\end{itemize}

For more information on \LaTeX\, see the uWaterloo Skills for the Academic Workplace  \href{https://uwaterloo.ca/information-systems-technology/services/electronic-thesis-preparation-and-submission-support/ethesis-guide/creating-pdf-version-your-thesis/creating-pdf-files-using-latex/latex-ethesis-and-large-documents}{course notes}. 
\footnote{
Note that while it is possible to include hyperlinks to external documents,
it is not wise to do so, since anything you can't control may change over time. 
It \emph{would} be appropriate and necessary to provide external links to 
additional resources for a multimedia ``enhanced'' thesis. 
But also note that if the \package{hyperref} package is not included, 
as for the print-optimized option in this thesis template, any \cmmd{href} 
commands in your logical document are no longer defined.
A work-around employed by this thesis template is to define a dummy \cmmd{href} 
command (which does nothing) in the preamble of the document, 
before the \package{hyperref} package is included. 
The dummy definition is then redifined by the
\package{hyperref} package when it is included.
}

The classic book by Leslie Lamport \cite{lamport.book}, author of \LaTeX , is worth a look too, and the many available add-on packages are described by 
Goossens \textit{et al} \cite{goossens.book}.

%----------------------------------------------------------------------
% END MATERIAL
% Bibliography, Appendices, Index, etc.
%----------------------------------------------------------------------

% Bibliography

% The following statement selects the style to use for references.  It controls the sort order of the entries in the bibliography and also the formatting for the in-text labels.
\bibliographystyle{plain}
% This specifies the location of the file containing the bibliographic information.  
% It assumes you're using BibTeX to manage your references (if not, why not?).
\cleardoublepage % This is needed if the book class is used, to place the anchor in the correct page,
                 % because the bibliography will start on its own page.
                 % Use \clearpage instead if the document class uses the "oneside" argument
\phantomsection  % With hyperref package, enables hyperlinking from the table of contents to bibliography             
% The following statement causes the title "References" to be used for the bibliography section:
\renewcommand*{\bibname}{References}

% Add the References to the Table of Contents
\addcontentsline{toc}{chapter}{\textbf{References}}

\bibliography{uw-ethesis}
% Tip 5: You can create multiple .bib files to organize your references. 
% Just list them all in the \bibliogaphy command, separated by commas (no spaces).

% The following statement causes the specified references to be added to the bibliography% even if they were not 
% cited in the text. The asterisk is a wildcard that causes all entries in the bibliographic database to be included (optional).
\nocite{*}
%----------------------------------------------------------------------

% Appendices

% The \appendix statement indicates the beginning of the appendices.
\appendix
% Add a title page before the appendices and a line in the Table of Contents
\chapter*{APPENDICES}
\addcontentsline{toc}{chapter}{APPENDICES}
% Appendices are just more chapters, with different labeling.
\chapter[PDF Plots From Matlab]{Matlab Code for Making a PDF Plot}
\label{AppendixA}
% Tip 4: Example of how to get a shorter chapter title for the Table of Contents 
%======================================================================
\section{Using the Graphical User Interface}
Properties of Matab plots can be adjusted from the plot window via a graphical interface. Under the Desktop menu in the Figure window, select the Property Editor. You may also want to check the Plot Browser and Figure Palette for more tools. To adjust properties of the axes, look under the Edit menu and select Axes Properties.

To set the figure size and to save as PDF or other file formats, click the Export Setup button in the figure Property Editor.

\section{From the Command Line} 
All figure properties can also be manipulated from the command line. Here's an example: 
\begin{verbatim}
x=[0:0.1:pi];
hold on % Plot multiple traces on one figure
plot(x,sin(x))
plot(x,cos(x),'--r')
plot(x,tan(x),'.-g')
title('Some Trig Functions Over 0 to \pi') % Note LaTeX markup!
legend('{\it sin}(x)','{\it cos}(x)','{\it tan}(x)')
hold off
set(gca,'Ylim',[-3 3]) % Adjust Y limits of "current axes"
set(gcf,'Units','inches') % Set figure size units of "current figure"
set(gcf,'Position',[0,0,6,4]) % Set figure width (6 in.) and height (4 in.)
cd n:\thesis\plots % Select where to save
print -dpdf plot.pdf % Save as PDF
\end{verbatim}

%----------------------------------------------------------------------
\end{document} % end of logical document
