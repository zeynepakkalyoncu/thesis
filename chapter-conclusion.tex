%======================================================================
\chapter{Conclusion and Future Work}
%======================================================================
\label{ch:conclusion}

In this thesis, we propose two innovations to successfully apply BERT to document retrieval with significant improvements on three TREC newswire collections:\ Robust04, Core17 and Core18.
First, to overcome the maximum input length restriction imposed by BERT, we focus on integrating sentence-level evidence to re-rank newswire documents.
This approach requires sentence-level relevance labels to train BERT for relevance prediction; however, relevance judgements in most test collections are provided only at the document level.
We address this challenge by leveraging sentence-level and passage-level relevance judgements fortuitously available in out-of-domain collections.
We fine-tune BERT with the goal of capturing cross-domain notions of relevance, which can be used to re-rank the longer documents in the newswire collections.

We show that relevance models learned with BERT can indeed be transferred across domains in a straightforward manner.
Combined with sentence-level relevance modeling, our simple model achieves state-of-the-art results across all three test collections.
Furthermore, our results suggest that judging only a small number of most relevant sentences in a document may be sufficient for effective document re-ranking.
Through our analyses we investigate the successes and failures of BERT in document re-ranking with respect to training data, query and document length, and semantic matching.
Our findings illustrate the relevance matching power of deep semantic information learned by BERT.
However, the analyses conducted in this thesis do not come close to fully understanding the effect of BERT in relevance matching.
A promising future direction based on our findings includes extended analyses of cross-domain transfer transfer and comparison of exact and semantic matching signals involved in re-ranking documents.