%======================================================================
\chapter{Conclusion and Future Work}
%======================================================================
\label{ch:conclusion}

In this thesis, we propose two innovations to successfully apply BERT to document retrieval with significant improvements on three TREC newswire collections:\ Robust04, Core17, and Core18.
To overcome the maximum input length restriction imposed by BERT, we focus on integrating sentence-level evidence to re-rank newswire documents.
This approach requires sentence-level relevance labels to train BERT for relevance prediction; however, relevance judgments in most test collections are provided only at the document level.
We address this challenge by leveraging sentence-level and passage-level relevance judgments fortuitously available in out-of-domain collections.
We fine-tune BERT with the goal of capturing cross-domain notions of relevance, which can be used to re-rank the longer documents in the newswire collections.

We show that relevance models learned with BERT can indeed be transferred across domains in a straightforward manner.
Combined with sentence-level relevance modeling, our simple model achieves state-of-the-art results across all three test collections.
Furthermore, we observe that judging only a small number of most relevant sentences in a document may be sufficient for effective document re-ranking.
Through our analyses we investigate the successes and failures of BERT in document re-ranking with respect to training data, query and document length, and semantic matching.
Our experiments suggest that both exact and semantic matching signals are indispensable for effective document ranking; ignoring either component leads to a notable decrease in effectiveness.
We also find that, in order to successfully transfer notions of relevance across domains, the sequence length of fine-tuning and evaluation data must be similar.

Our findings illustrate the deep relevance matching power enabled by BERT.
However, the analyses conducted in this thesis do not come close to fully understanding the effect of BERT in relevance matching.
In future work, we aim to delve into a deeper study of BERT attention by visualizing multiple attention heads across multiple layers for all query terms.
By examining the entropy of attention distribution for different models, we hope to gain a better understanding of the matching tendencies of BERT.
A natural extension to this work would be to replace BERT with other language models such as RoBERTa~\cite{DBLP:journals/corr/abs-1907-11692} or XLNet~\cite{DBLP:journals/corr/abs-1906-08237} to see how different model hyperparameters or training objectives might affect relevance matching behavior.
%We believe cross-domain relevance transfer works 
To further investigate the contribution of semantic matching with BERT, we also intend to repeat our ranking experiments by relying solely on sentence scores in the future.
Another promising future direction based on our findings includes extending our efforts to other domains to investigate various other factors that influence cross-domain relevance transfer.
Finally, it might be worthwhile to explore different pruning or distillation techniques to manage the latency and memory requirements of our system.