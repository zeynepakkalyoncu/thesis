%======================================================================
\chapter{Cross-Domain Relevance Transfer with BERT}
%======================================================================

\section{Datasets}

\myworries{Add statistics and examples}

\myworries{Elaborate on splits}

\myworries{Add motivation, gist of each dataset}

\myworries{Add length distribution}

\subsection{Fine-Tuning}

As discussed in Section \ref{intro}, applying BERT to document retrieval requires leveraging passage- or sentence-level relevance judgements fortuitously available in large text collections.
Since no such newswire collection currently exists, we train the BERT relevance classifier on three out-of-domain collections.

\subsubsection{TREC Microblog (MB)}

TREC Microblog datasets draw from the Microblog Tracks at TREC from 2011 to 2014, with topics (i.e., queries) and relevance judgments over tweets.
We use the data prepared by Rao et al. \cite{rao2019tweet}.
We extract the queries, tweets and relevance judgements from the dataset, excluding metadata such as query time and URLs of the tweets.
Relevance judgements in Microblog are in fact reported in a three-point scale where (``irrelevant'', ``relevant'' and ``highly relevant''); however, for the purposes of this work we treat both degrees of relevance as equal.
We tokenize both query and tweets...

\begin{figure}[b!]
	\begin{framed}
		\centering
    		\textbf{Query:} bbc world service staff cuts \\
    		\textbf{Text:} irish times : bbc world service confirms cuts : the bbc world service will shed around 650 jobs or more \\
    		\textbf{Relevance:} 1 (``relevant'')
	\end{framed}
\label{mb-example}
 \caption{An example of... MB}
\end{figure}

We \myworries{sample} 25\% of the data for the validation set, and use the rest for fine-tuning BERT.
\myworries{What else?}

\begin{table}[t]
\vspace{0.2cm}
\centering
\begin{tabular}{llll}
\toprule
\textbf{Type} \mbox{\hspace{0.5cm}} & \textbf{Training Set} \mbox{\hspace{1.0cm}} & \textbf{Validation Set} \mbox{\hspace{1.0cm}} & \textbf{Total} \mbox{\hspace{1.0cm}} \\
\toprule
Number of queries & 166 & 59 & 225 \\
Number  of tweets & asd & asd & asd \\
Percentage of relevant tweets  & asd      & asd & asd   \\
\bottomrule
\end{tabular}
\vspace{0.2cm}
\caption{...}% at the $p<0.01$ level.}
\label{tab:mb-stats}
\vspace{-0.6cm}
\end{table}

\myworries{Add total}

\subsubsection{MicroSoft MAchine Reading Comprehension (MS MARCO)}

MS MARCO \cite{nguyen2016msmarco} features user queries sampled from Bing’s search logs and passages extracted from web documents.
The dataset is composed of tuples of a query with relevant and non-relevant passages.
Each query is associated with sparse relevance judgments by human editors.
In other words, some queries may have no relevant passage.
Because the dataset was initially built from top 10 passages from the Bing search engine and then annotated, some relevant passages might not have been retrieved with BM25.
On average, however, each query is associated with one relevant passage.

The models in Section \myworries{ref} were trained on less than 2\% of the total training set due to the size of the dataset and time required to train on it even on TPUs.
According to Nogueira et al. \cite{nogueira2019passage}, training for up to 12.5\% of the total data does not improve MRR@10 on the validation set.

\begin{table}[b]
\vspace{0.2cm}
\centering
\begin{tabular}{llll}
\toprule
\textbf{Type} \mbox{\hspace{0.5cm}} & \textbf{Training Set} \mbox{\hspace{1.0cm}} & \textbf{Validation Set} \mbox{\hspace{1.0cm}} & \textbf{Total} \mbox{\hspace{1.0cm}} \\
\toprule
Number of queries & 12.8M & asd & asd \\
Number  of ? & asd & asd & asd \\
Percentage of relevant passages  & asd      & asd & asd   \\
\bottomrule
\end{tabular}
\vspace{0.2cm}
\caption{...}% at the $p<0.01$ level.}
\label{tab:marco-stats}
\vspace{-0.6cm}
\end{table}

\myworries{TODO}
\begin{figure}[b!]
	\begin{framed}
		\centering
    		\textbf{Query:} bbc world service staff cuts \\
    		\textbf{Text:} irish times : bbc world service confirms cuts : the bbc world service will shed around 650 jobs or more \\
    		\textbf{Relevance:} 1 (``relevant'')
	\end{framed}
\label{marco-example}
 \caption{An example of... MS MARCO}
\end{figure}

\subsubsection{TREC Complex Answer Retrieval (CAR)}

TREC CAR \cite{dietz2017trec} uses queries and paragraphs extracted from English Wikipedia: each query is formed by concatenating an article title and a section heading, and passages in that section are considered relevant.
This makes CAR, essentially, a synthetic dataset.
...

\begin{table}[b]
\vspace{0.2cm}
\centering
\begin{tabular}{llll}
\toprule
\textbf{Type} \mbox{\hspace{0.5cm}} & \textbf{Training Set} \mbox{\hspace{1.0cm}} & \textbf{Validation Set} \mbox{\hspace{1.0cm}} & \textbf{Total} \mbox{\hspace{1.0cm}} \\
\toprule
Number of queries & 12.8M & asd & asd \\
Number  of ? & asd & asd & asd \\
Percentage of relevant passages  & asd      & asd & asd   \\
\bottomrule
\end{tabular}
\vspace{0.2cm}
\caption{...}% at the $p<0.01$ level.}
\label{tab:car-stats}
\vspace{-0.6cm}
\end{table}

\myworries{TODO}
\begin{figure}[b!]
	\begin{framed}
		\centering
    		\textbf{Query:} bbc world service staff cuts \\
    		\textbf{Text:} irish times : bbc world service confirms cuts : the bbc world service will shed around 650 jobs or more \\
    		\textbf{Relevance:} 1 (``relevant'')
	\end{framed}
\label{car-example}
 \caption{An example of... CAR}
\end{figure}

\subsection{Evaluation}

We conduct end-to-end document ranking experiments on three TREC newswire collections:\
the Robust Track from 2004 (Robust04) and the Common Core Tracks from 2017 and 2018 (Core17 and Core18).

\subsubsection{Robust04}

Robust04 comprises 250 topics, with relevance judgments on a collection of 500K documents (TREC Disks 4 and 5 except the Congressional Record).
We apply five-fold cross-validation \cite{} on...

\subsubsection{Core17 \& Core18}
 
Core17 and Core18 have only 50 topics each; the former uses 1.8M articles from the New York Times Annotated Corpus while the latter uses around 600K articles from the TREC Washington Post Corpus.
\myworries{Motivation for this dataset?}
Similar to Robust04, we split the 50 topics into ...