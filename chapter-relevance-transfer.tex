%======================================================================
\chapter{Cross-Domain Relevance Transfer with BERT}
%======================================================================
\label{ch:model}

%\myworries{Go into details of BERT, then how we use it...}
%\myworries{Maybe give example for how BERT for relevance modeling looks like? Input? May draw my own diagram}
%\myworries{Can I add some math-y stuff to describe what BERT is doing?}

Our proposed model is based on sentence-level relevance modeling and document re-ranking with BERT.
By training BERT as a relevance classifier, we aim to extract valuable semantic matching signals which can be leveraged to re-rank a list of candidate documents retrieved with a term-matching technique such as BM25.
We also explore applying cross-domain relevance transfer to exploit models of relevance learned on out-of-domain collections, which is crucial in re-ranking documents that are too long for BERT to directly process.
This chapter introduces the datasets that we use and, and describes the details of our BERT-based sentence-level relevance classifier and document re-ranker.

\section{Modeling Relevance with BERT}

We propose modeling sentence-level and passage-level relevance with BERT to capture semantic signals helpful for relevance prediction.
%\myworries{Better alternative to capture semantic signals?}
This approach is motivated by the application of transfer learning in NLP where a large transformer model trained for language modeling can be used for various downstream tasks.
In our implementation, we choose BERT as our base mode.
BERT is trained on copious amounts of unsupervised data from the Google BookCorpus and English Wikipedia with masked language modeling.
%\myworries{Talk about relevance prediction too}
Although the training procedure doesn't involve any explicit objective to extract linguistic features, it has been shown to implicitly recognize such features as subject-verb agreement and conference resolution~\cite{jawahar2019does, clark2019does}, which allows a number of NLP tasks to greatly benefit from features implicitly encoded in BERT weights.

\subsection{Relevance Classifier}

The core of our model is a BERT-based sentence-level relevance classifier.
In other words, we build a model on top of BERT to predict a relevance score $ s_i $ for a sentence or passage $ d_i $ given a query $ q $.
Because the maximum input length that BERT can handle is 512 tokens, we limit our training data to sentence-level and passage-level datasets.
In other words, $ d_i $ are either tweets drawn from TREC Microblog or passages from MS MARCO or TREC CAR.
Following Nogueira et al.~\cite{nogueira2019passage}, we frame relevance modeling as a binary classification task.
%Figure \myworries{X} illustrates how BERT can be used for to predict the relevance of a given sentence.
%\myworries{Add diagram to explain the relevance modeling process}
More specifically, we feed query-text pairs into the BERT model with their respective relevance judgements (i.e., 0 for non-relevant and 1 for relevant).
Through this training process BERT learns to assign a relevance score to unseen text for a given query.
The details of the input representation to BERT and specifics of fine-tuning BERT for relevance prediction are discussed at length in the remainder of this chapter.

\subsubsection{Input Representation}

%\begin{figure}[t!]
%	\begin{framed}
%		\centering
%		blahblahblah
%	\end{framed}
%\label{bert-tokenization-example}
% \caption{\myworries{Put tokenized BERT input example?}}
%\end{figure}

\begin{figure}[b!]
\centering
  \includegraphics[width=6.5in]{bert_input.png}
\caption{Illustration of BERT input representation adapted from Devlin et al.~\cite{devlin2018bert}.}
\label{fig:bert_input}
\end{figure}

We form the input to BERT by concatenating the query $ q $ and a sentence $ d $ into the sequence [\texttt{[CLS]}, $q$, \texttt{[SEP]}, $d$, \texttt{[SEP]}] .
The \texttt{[SEP]} metatoken is used to distinguish between two non-consecutive token sequences in the input, i.e., query and text, and the \texttt{[CLS]} signifies a special symbol for classification output.
Although BERT supports variable length token sequences, the final input length must be consistent across each batch.
Therefore, we pad each sequence in a mini-batch to the maximum length in the batch.

The complete input embeddings of BERT is comprised of token, segmentation, and position embeddings.
The first is constructed by tokenizing the above sequence with the proper metatokens in place with the BERT tokenizer.
Since BERT was trained based on WordPiece tokenization, we use the same tokenizer to achieve optimal performance.
WordPiece tokenization may break words into multiple subwords in order to more efficiently deal with out-of-vocabulary words and better represent complex words.
During training, the subwords derived with WordPiece tokenization are reconstructed based on the training corpus.
After tokenization, each token in the input sequence is converted into token IDs corresponding to the index in BERT's vocabulary.
Tokens that do not exist in the vocabulary are represented with a special \texttt{[UNK]} token.

The segment embeddings indicate the start and end of each sequence, whether it be a single sequence or a pair.
For relevance classification where we have two texts in the input sequence, i.e., query and sentence, the segment embeddings corresponding to the tokens of the first sequence, i.e., the query, are all 0's, and those for the second sequence, i.e., the document, are all 1's.
The position embeddings are learned for sequences up to 512 tokens, and help BERT recognize the relative position of each token in the sequence.
The input representation for a sample short query-sentence pair is shown in Figure \ref{fig:bert_input}.

%An example BERT input for MB is shown in Figure \myworries{X}.
%\myworries{Example with complex words}

\subsubsection{Fine-Tuning}

%\myworries{Go more into detail about NN stuff?}
A variety of useful deep semantic features are already encoded in pretrained BERT weights.
It is thus possible to fine-tune BERT for a specific downstream task with less data and time by adding a fully-connected layer on top of the network.
Intuitively, the lower layers of the network have already been trained to capture latent features relevant to the task.

To fine-tune BERT for relevance modeling, we add a single layer neural network on top of BERT for classification.
This layer consists of $ K \times H $ randomly initialized neurons where $ K $ is the number of classifier labels and $ H$ is the hidden state size.
For relevance classification, we have two labels indicating whether the sentence is relevant or non-relevant for the given query ($ K = 2 $).

The final hidden state corresponding to the first token, i.e., \texttt{[CLS]}, provides a $ H $-dimensional aggregate representation of the input sequence that can be used for classification.
We feed the final hidden state in the model corresponding to \texttt{[CLS]} into the classification layer.
The probability that the sentence $ d_i $ is relevant to the query $ q_i $ is thus computed with standard softmax:
 
\begin{equation}
\sigma (y_i) = \frac{e^{y_i}}{\sum_j e^{y_j}}
\end{equation}

\noindent where $\sigma (y_i)$ maps the arbitrary real value $ y_i $ into a probability distribution.
Intuitively, $\sigma (y_i)$ represents the relevance score for the sentence $ d_i $.
The parameters of BERT and the additional softmax layer are optimized jointly to maximize the log-probability of the correct label with cross-entropy loss.

\section{Reranking with BERT}

%\begin{figure}[b!]
%\centering
%  \includegraphics[width=4in]{bert_real.png}
%\caption{...}
%\label{fig:bert_real}
%\end{figure}

Fine-tuning BERT on relevance judgements of query-text pairs allows us to obtain a model of relevance so that we can compute sentence-level relevance scores easily on any collection.
However, recall that we train BERT on sentence-level or passage-level datasets so as not to exceed the maximum input size of BERT.
These training datasets come from very different distributions than the test collections introduced in Chapter~\ref{ch:results}.
In order to predict relevance on much longer newswire documents, we explore cross-domain relevance transfer by using models trained on MB, MS MARCO and CAR on newswire collections.
Our hypothesis is that if a neural network with a large capacity such as BERT can capture relevance in one domain, the model of relevance might successfully transfer to other domains.
%\myworries{Probably need a couple more sentences to make this more coherent}

To apply cross-domain relevance transfer, we retrieve relevant documents from the collection to depth 1000 with BM25 and split each document into its constituent sentences to match the input size of BERT.
We then run inference over the sentences with our models fine-tuned on out-of-domain datasets to compute a score for each sentence.
We determine overall document scores by combining exact and semantic matching signals.
Based on BM25+RM3 document scores we know a ranking of documents based on exact matches of query terms.
Sentence-level scores obtained with BERT reveal other implicit semantic information not evident to BM25.
By combining the two sets of relevance matching signals, we establish a more diverse notion of relevance, leading to a better ranking of documents.
%\myworries{Give example diagram of sentence score ranking? Maybe with BM25?}

%Using either set of scores to rank documents neglects crucial information from the other, so we interpolate the scores 
Therefore, to determine overall document relevance, we combine the top $ n $ scores with the original document score as follows:
\begin{equation} \label{eq:1}
s_f = a \cdot s_{doc}  + (1 - \alpha) \cdot \sum_{i = 1}^n w_i \cdot s_i
\end{equation}

\noindent where $ s_{doc} $ is the original document score and $ s_i $ is the $ i $-th top scoring sentence according to BERT.
In other words, the relevance score of a document comes from the combination of a document-level term-matching score and relevance evidence contributions from the top sentences in the document as determined by BERT.
The parameters $ \alpha $ and the $ w_i $'s in Equation~\ref{eq:1} can be tuned via cross-validation.
%\myworries{Motivation?}

\section{Experimental Setup}

\subsection{Training and Inference with BERT}

We fine-tune $ \textrm{BERT}_{\textrm{\scriptsize Large}} $~\cite{devlin2018bert} on the datasets introduced earlier in this section:\ TREC Microblog, MS MARCO, and TREC CAR.
In our implementation we adopt $ \textrm{BERT}_{\textrm{\scriptsize Large}} $'s \texttt{BertForNextSentencePrediction} interface from the Huggingface \texttt{transformers} (previously known as \texttt{pytorch-pretrained-bert}) library\footnote{https://github.com/huggingface/transformers} as our base model.
The maximum sequence length, i.e., 512 tokens, is used for BERT in all our experiments.

The fine-tuning procedure introduces few new hyperparameters in addition to those already used in pre-training:\ batch size, learning rate, and number of training epochs.
Due to the large amount of training data in MS MARCO and TREC CAR, BERT is initially trained on Google's TPU's with a batch size of 32 for 400k iterations.
% following ~\cite{nogueira2019passage}.
We use Adam \cite{kingma2014adam} with an initial learning rate of $ 3 \times 10^{-6}$, $ \beta_1 = 0.9 $ and $ \beta_2 = 0.999 $ and L2 decay of 0.01.
Learning rate warmup is applied over the first 10k steps with linear decay of learning rate.
We apply dropout with probability of 0.1 across all layers.

We train all other models using cross-entropy loss for 5 epochs with a batch size of 16.
We conduct all our experiments on NVIDIA Tesla P40 GPUs with PyTorch v1.2.0.
We use Adam \cite{kingma2014adam} with an initial learning rate of $ 1 \times 10^{-5}$, linear learning rate warmup at a rate of 0.1 and decay of 0.1.
We find that applying diminishing learning rates is especially important in fine-tuning BERT in order to preserve the information encoded in the original BERT weights and speed up training.

\subsection{Evaluation}

We retrieve an initial ranking of 1000 documents for each query in Robust04, Core17 and Core18 using the open-source Anserini information retrieval toolkit based on Lucene 8.
To ensure fairness across all three collections, we use BM25 with RM3 query expansion with default parameters.
Before running inference with BERT to obtain relevance scores, we preprocess the retrieved documents:\
First, we clean the documents by stripping any HTML/XML tags and split each document into its constituent sentences with NLTK's Stanford Tokenizer\footnote{https://nlp.stanford.edu/software/tokenizer.shtml}.
If the length of a sentence with the meta-tokens still exceeds the maximum input length of BERT, we further segment the spans into fixed sized chunks.

%Following the procedure in Section \myworries{X}, we obtain a relevance score for each sentence.
We experiment with the number of top scoring sentences to consider while computing the overall score, and find that using only the top 3 sentences is often enough.
In general, considering any more doesn't yield better results.
To tune hyperparameters in Equation~\ref{eq:1}, we apply five-fold cross-validation over TREC topics.
For Robust04, we follow the five-fold cross-validation settings in Lin~\cite{lin2019neural} over 250 topics.
For Core17 and Core18 we similarly apply five-fold cross validation.
We learn parameters $\alpha$ and the $w_i$ on four folds via exhaustive grid search with $ w_1 = 1 $ and varying $ a, w_2, w_3 \in [0, 1] $ with a step size 0.1, selecting the values that yield the highest AP on the remaining fold.
We report retrieval effectiveness in terms of AP, P@20, and NDCG@20.